%%PREAMBLE STARTS HERE%%
\documentclass[brazilian,10.7pt,a4paper]{article}
\usepackage[a4paper]{geometry}
\usepackage[brazil]{babel}
\usepackage[utf8]{inputenc}
\usepackage{graphicx}
\usepackage{amsmath}
\usepackage{amsfonts}
\usepackage{amssymb}
\usepackage{multicol}
\usepackage{lmodern}

\geometry{verbose,tmargin=2.6cm,bmargin=2.6cm,lmargin=1.3cm,rmargin=1.9cm}
\pagestyle{empty}

%Title information%
\title{\textbf{Comentários sobre o artigo ``Eletrostática em sistemas coloidais carregados"}\\
\vspace{5pt}\small{\textbf{Comments about the paper Electrostatics in charged colloidal systems}}\vspace{-3pt}}
\author{A.V. Andrade-Neto\footnote{Endereço de correspondência: aneto@uefs.br.}\hspace{5pt}\textsuperscript{1}\vspace{7pt}\\ 
\small{\textsuperscript{1}Universidade Estadual de Feira de Santana, Departamento de Física, Feira de Santana, BA, Brasil}\vspace{6pt}}
\date{\small{Recebido em 25 de Agosto, 2017. Revisado em 26 de Outubro, 2018. Aceito em 03 de Novembro, 2018.}\vspace{-45pt}}
%%%%%%%%%

%%PREAMBLE ENDS HERE%%
\begin{document}
\renewcommand{\abstractname}{}
\frenchspacing
\maketitle
\begin{abstract}
\small{
Comentam-se nessas notas resultados obtidos em trabalho publicado neste periódico sobre a eletrostática em
sistemais coloidais, Revista Brasileira de Ensino de Física \textbf{40}, e5408 (2018).\\
\textbf{Palavras-chave:} eletrostática, potencial blindado\\
\par
Some remarks about results of an article published in this journal, Revista Brasileira de Ensino de Física \textbf{40},
e5408 (2018), concerning electrostatics applied to colloidal systems are presented.\\
\textbf{Keywords:} electrostatic, screening.}\\
\end{abstract}
\begin{multicols}{2}
No artigo intitulado “Eletrostática em sistemas coloidais carregados” [1], é apresentado “um estudo da eletrostática aplicada aos sitemas coloidais” com o objetivo de possibilitar “uma visão mais ampla dos conceitos estudados em sala através de uma aplicação direta da teoria”. Ainda que pertinente, por aplicar conceitos familiares da eletrostática a uma siuação de interesse, o trabalho apresenta alguns problemas conceituais que discutiremos nessas notas. Apresentaremos uma crítica a maneira como na Referência [1] são derivadas equações
da eletrostática para um problema específico (sistemas coloidais) e como essas [equações (41) e (44) daquele trabalho] são interpretadas como uma possível nova lei de Gauss. Indo direto ao ponto, são apresentadas expressões que, segundo os autores, seriam “equações equivalentes para a lei de Gauss, tanto na forma integral como na forma diferencial”. Conforme mostraremos, trata-se apenas das equações básicas da eletrostática escritas para uma situação específica ou escritas de forma equivocada. \par
Vamos começar fazendo uma breve revisão da eletrostática, cujas leis fundamentais (equações de Maxwell)
para um meio dielétrico são dadas por \\
\begin{equation}\label{1}
\nabla\times\vec{E}(\vec{r}) = 0
\end{equation}
\vspace{1pt}
\begin{equation}\label{2}
\nabla\cdot\vec{D}(\vec{r}) = \rho_{ext}(\vec{r})\;, 
\end{equation}
onde $\vec{E}$ e $\vec{D}$ são, respectivamente, o campo elétrico e o vetor deslocamento e $\rho_{ext}$ é a densidade de carga externa. \\
Em termos do vetor campo elétrico $\vec{E}$, a equação \eqref{2} é escrita como \\
\\
\\
\begin{equation}\label{3}
\nabla\cdot\vec{E}(\vec{r})=\frac{\rho_{total}(\vec{r})}{\epsilon_{0}}=\frac{\rho_{ext}(\vec{r})+\rho_{pol}(\vec{r})}{\epsilon_{0}}\;,
\end{equation}
onde $\epsilon_{0}$ é a permissividade elétrica do vácuo, $\rho_{total} = \rho_{ext} + \rho_{pol}$ é a densidade de carga total do sistema e $\rho_{pol}$ é a densidade de carga de polarização dada por \;$\rho_{\large{P}} = -\nabla\cdot\vec{P}$ onde $\vec{P}$ é o vetor polarização elétrica. Para muitos materiais há uma
relação simples entre $\vec{E}$ e $\vec{D}$ dada por
\\
\begin{equation}\label{4}
\vec{D} = \epsilon\vec{E}\;,
\end{equation}
onde $\epsilon$ é a permissividade elétrica do material e no caso geral é uma função da posição e do tempo \ref{2}. Se considerarmos $\epsilon$ uma constante característica do material a equação \eqref{2} se torna
\\
\begin{equation}\label{5}
\nabla\cdot\vec{E}(\vec{r})=\frac{\rho_{ext}(\vec{r})}{\epsilon}\;.
\end{equation}
\par As equações \eqref{1} e \eqref{5}, com condições de contorno apropriadas, determinam completamente o campo elétrico \ref{3}.\\
Apenas para sistemas suficientemente simétricos, é possível determinar o campo elétrico utilizando somente a lei de gauss, que é a situação apresentada em todos os livros textos de eletromagnetismo [4]- [7]. A equação \eqref{1} expressa o fato de que o campo eletrostático é conservativo, podendo, então, ser expresso como o gradiente de um campo escalar, i.e., %corrigir referencias do final do texto
\\
\begin{equation}\label{6}
\vec{E}(\vec{r})= -\nabla\varphi(r)\;.
\end{equation}
\par Isso permite combinar as equações vetoriais \eqref{1} e \eqref{5} em uma única equação diferencial escalar. Utilizando \eqref{6} em \eqref{5} obtemos a equação de Poisson
\\
\begin{equation}\label{7}
\nabla^{2}\varphi(r)=-\frac{\rho_{ext}(\vec{r})}{\epsilon}
\end{equation}
A princípio, o potencial elétrico pode ser determinado via equação de Poisson. Feito isso, podemos em seguida calcular o campo elétrico correspondente através da equação \eqref{6}. Esse foi o procedimento realizado na referência 1.
\par Vamos inicialmente fazer alguns comentários sobre o modelo adotado. A primeira coisa a ser salientada é que o sistema original é constituído de muitas partículas eletricamente carregadas que interagem entre si via forças coulombianas, constituindo, assim, um problema de muitos corpos de difícil solução. É necessário, então, adotar um modelo que torne o sistema tratável, do ponto de vista matemático, e ao mesmo tempo preserve suas características físicas essenciais. É admitido que o sistema é globalmente neutro, i.e, os números de cargas positivas e negativas são iguais. Esse tipo de sistema é comum em variadas áreas da física. Além de estruturas coloidais ele é encontrado em plasmas, em metais e semicondutores dopados [8]. Uma maneira de abordar sistemas desse tipo é considerar uma carga positiva $Q$ mais uma nuvem de partículas de cargas $-q$ ao seu redor. As partículas negativas tendem a se concentrar em torno da carga positiva, reduzindo assim o campo elétrico criado pela carga $Q$. Esse fenômeno é conhecido como blindagem eletrostática [9].
\par Para esse modelo, a densidade de carga pode ser escrita como
\\
\begin{equation}\label{8}
\rho_{ext}(r)=Q\delta(\vec{r})-qn(\vec{r})
\end{equation}
onde $\delta(\vec{r})$ é a função delta de Dirac e $n(\vec{r})$ é a densidade de partículas negativas (número de partículas por unidade de volume). Utilizando \eqref{8} em \eqref{7} temos que
\\
\begin{equation}\label{9}
\nabla^{2}\varphi(r)=-\frac{1}{\epsilon}[Q\delta(\vec{r})-qn(\vec{r})]
\end{equation}
\par A equação \eqref{9} é de difícil solução, mesmo porque a concentração $n(r)$ não é conhecida. Para prosseguir é necessário uma nova hipótese que relacione $n(r)$ ao potencial eletrostático $\varphi(r)$. Os modelos de Debye-Hückel e Thomas-Fermi admitem que $qn$ é proporcional ao potencial $\varphi$
\\
\begin{equation}\label{10}
qn(\vec{r})\approx \epsilon k^{2}_{0}\varphi(r)
\end{equation}
onde $k_{0}$ é o inverso do comprimento de blindagem e para o modelo de Debbye-Hückel temos que
\\
\begin{equation}\label{11}
k^{2}_{0}=k^{2}_{DH}=\frac{q^{2}n}{\epsilon k_{B}T}
\end{equation}
onde $k_{B}$ é a constante de Boltzmann e T é a temperatura absoluta. Para o modelo de Thomas-Fermi temos que
\\
\begin{equation}\label{12}
k^{2}_{0}=k^{2}_{TF}=\frac{mq^{2}K_{F}}{\epsilon\pi^{2}\hbar^{2}}
\end{equation}
onde $m$ é a massa das partículas, $K_{F}$ é o vetor de onda de Fermi e $\hbar$ é a constante reduzida de Planck. Usando \eqref{10} em \eqref{9} temos que
\\
\begin{equation}\label{13}
[\nabla^{2}-k^{2}_{0}]\varphi(r)=-\frac{Q}{\epsilon}\delta(\vec{r})
\end{equation}
a qual, a menos do fator delta do lado direito (que é a forma correta de expressar a equação do modelo considerado), é semelhante à equação (45) da referência \cite{ramos}. Deve ser enfatizado que a hipótese \eqref{9} é fundamental para se obter a equação \eqref{13}, em outras palavras, essa equação nada mais é que a equação de Poisson escrita para um caso específico e não, como pode transparecer da referência \cite{ramos}, uma nova equação equivalente à equação de Poisson.
\par A solução da \eqref{13} pode ser obtida pelo método das funções de Green e é dada por 
\\
\begin{equation}\label{14}
\varphi(r)=\frac{Q}{4\pi\epsilon}\frac{e^{k_{o}r}}{r}\;,
\end{equation}
que é denominado potencial eletrostático blindado e é a equação (22) da referência \cite{ramos}. Segundo os autores, essa equação "nos fornece o potencial elétrico efetivamente gerado por uma partícula carregada nesse modelo de sistema coloidal". Essa afirmação precisa ser melhor esclarecida. Na verdade, a distribuição de carga que gera o potencial dado pela equação \eqref{14} é uma carga pontual $Q$ mais a nuvem de carga negativa que a envolve [ver equação \eqref{20} adiante]. Desse modo, podemos considerar a partícula de carga $Q$ mais a nuvem negativa que a envolve como uma nova "partícula"(ou quasipartícula na linguagem da física da matéria condensada \cite{ribeiro}). Deve-se destacar que a carga associada a essa quasipartícula depende da extensão da dupla camada elétrica formada, relacionada ao parâmetro $k_{0}$ [ver equação \eqref{21} adiante]. Conhecido o potencial eltrostático, o campo elétrico pode ser facilmente calculado. tomando o gradiente da equação \eqref{14} encontramos
\\
\begin{equation}\label{15}
\vec{E}(r)=\frac{Q}{4\pi\epsilon}(1+k_{0}r)\frac{e^{k_{0}r}}{r^{2}}\uvec{r}
\end{equation}
que é a equação (26) da referência \cite{ramos}. Contudo, é importante enfatizar que essa expressão não corresponde ao campo elétrico gerado por uma carga pontual, mas sim por uma carga $Q$ mais a nuvem negativa que a envolve, ou seja, um campo elétrico blindado. A densidade de carga associada ao campo elétrico descrito pela equação \eqref{15} pode ser calculado da seguinte forma. Fazendo $f(r)=e^{-k_{0}r}$ e $g(r) = 1/r$ e usando a identidade
\\
\begin{equation}\label{16}
\nabla^{2}(fg)=f(\nabla^{2}g)+g(\nabla^{2}f)+2(\nabla f)\cdot(\nabla g)
\end{equation}
obtemos
\\
\begin{equation}\label{17}
\nabla^{2}(\frac{e^{-k_{0}r}}{r}=\frac{k^{2}_{0}e^{-k_{o}r}}{r}-4\pi e^{-k_{0}r}\delta(r))
\end{equation}
onde usamos que
\\
\begin{equation}\label{18}
\nabla^{2}(\frac{1}{r})=-4\pi\delta(r)\;.
\end{equation}
\par Utilizando as equações \eqref{14} e \eqref{17} na equação \eqref{7} obtemos
\\
\begin{equation}\label{19}
\rho_{ext}(r)=Q[\delta(r)-\frac{k^{2}_{0}}{4\pi r}]e^{-k_{0}r}\;.
\end{equation}
\par Podemos agora

\begin{thebibliography}{10}
\bibitem{ramos}
I.R.O. Ramos, J.P.M. Braga, J.V.A. Ataíde, A.P. Lima and L. Holanda, Revista Brasileira de Ensino de Física \textbf{40}, e5408 (2018).
\bibitem{andrade}
A.V Andrade-Neto, Revista Brasileira de Ensino de Física \textbf{39}, e2304 (2017).
\bibitem{tort}
A.C. Tort, Revista Brasileira de Ensino de Física \textbf{33}, 2701 (2011).
\bibitem{hm}
H.M. Nussenzveig, \textit{Curso de Física Básica} (Edgar Blücher, São Paulo 1997), v. 3.
\bibitem{reitz}
J.R Reitz, F.J. Milford e R.W. Chisty, \textit{Fundamentos da Teoria Eletromagnética} (Editora Campus, Rio de Janeiro, 1982)
\bibitem{dj}
D.J. Griffiths, \textit{Introduction to Eletrodynamics} (Pearson Education, Boston, 2013), 4\textsuperscript{\underline{a}} ed.
\bibitem{jackson}
J.D. Jackson, \textit{Classical Electrodynamics} (John Wiley Sons, New Jersey, 1999).
\bibitem{platzman}
P.M. Platzman and P.A. Wolff, \textit{Waves and Interactions in Solid State Plasmas} (Academic Press, New York, 1973).
\bibitem{kittel}
C. Kittel, \textit{Introdução à Física do Estado Sólido} (LTC Editora, Rio de Janeiro, 2006), 8\textsuperscript{\underline{a}} ed., p. 341.
\bibitem{ribeiro}
A.V. Andrade-Neto, A. Ribeiro Neto and A. Jorio, Revista Brasileira de Ensino de Física \textbf{39}, e3310 (2017).
\end{thebibliography}
\end{multicols}
\end{document}
