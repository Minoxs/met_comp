%%PREAMBLE STARTS HERE%%
\documentclass[brazilian,10.7pt,a4paper]{article}
\usepackage[a4paper]{geometry}
\usepackage[brazil]{babel}
\usepackage[utf8]{inputenc}
\usepackage{graphicx}
\usepackage{amsmath}
\usepackage{amsfonts}
\usepackage{amssymb}
\usepackage{multicol}
\usepackage{lmodern}

\geometry{verbose,tmargin=2.6cm,bmargin=2.6cm,lmargin=1.3cm,rmargin=1.9cm}
\pagestyle{empty}

%Title information%
\title{\textbf{Comentários sobre o artigo ``Eletrostática em sistemas coloidais carregados"}\\
\vspace{5pt}\small{\textbf{Comments about the paper Electrostatics in charged colloidal systems}}\vspace{-3pt}}
\author{A.V. Andrade-Neto\footnote{Endereço de correspondência: aneto@uefs.br.}\hspace{5pt}\textsuperscript{1}\vspace{7pt}\\ 
\small{\textsuperscript{1}Universidade Estadual de Feira de Santana, Departamento de Física, Feira de Santana, BA, Brasil}\vspace{6pt}}
\date{\small{Recebido em 25 de Agosto, 2017. Revisado em 26 de Outubro, 2018. Aceito em 03 de Novembro, 2018.}\vspace{-34pt}}
%%%%%%%%%

%%PREAMBLE ENDS HERE%%
\begin{document}
\renewcommand{\abstractname}{}
\frenchspacing
\maketitle
\begin{abstract}
\small{
Comentam-se nessas notas resultados obtidos em trabalho publicado neste periódico sobre a eletrostática em
sistemais coloidais, Revista Brasileira de Ensino de Física \textbf{40}, e5408 (2018).\\
\textbf{Palavras-chave:} eletrostática, potencial blindado\\
\par
Some remarks about results of an article published in this journal, Revista Brasileira de Ensino de Física \textbf{40},
e5408 (2018), concerning electrostatics applied to colloidal systems are presented.\\
\textbf{Keywords:} electrostatic, screening.}\\
\end{abstract}
\begin{multicols}{2}
No artigo intitulado “Eletrostática em sistemas coloidais carregados” [1], é apresentado “um estudo da eletrostática aplicada aos sitemas coloidais” com o objetivo de possibilitar “uma visão mais ampla dos conceitos estudados em sala através de uma aplicação direta da teoria”. Ainda que pertinente, por aplicar conceitos familiares da eletrostática a uma siuação de interesse, o trabalho apresenta alguns problemas conceituais que discutiremos nessas notas. Apresentaremos uma crítica a maneira como na Referência [1] são derivadas equações
da eletrostática para um problema específico (sistemas coloidais) e como essas [equações (41) e (44) daquele trabalho] são interpretadas como uma possível nova lei de Gauss. Indo direto ao ponto, são apresentadas expressões que, segundo os autores, seriam “equações equivalentes para a lei de Gauss, tanto na forma integral como na forma diferencial”. Conforme mostraremos, trata-se apenas das equações básicas da eletrostática escritas para uma situação específica ou escritas de forma equivocada.\\
Vamos começar fazendo uma breve revisão da eletrostática, cujas leis fundamentais (equações de Maxwell)
para um meio dielétrico são dadas por
\begin{equation}\label{1}
\nabla\times\vec{E}(\vec{r}) = 0
\end{equation}
\vspace{5.25pt}
\begin{equation}\label{2}
\nabla\cdot\vec{D}(\vec{r}) = \rho_{ext}(\vec{r}) , 
\end{equation}
onde $\vec{E}$ e $\vec{D}$ são, respectivamente, o campo elétrico e o vetor deslocamento e $\rho_{ext}(\vec{r})$ é a densidade de carga externa. Em termos do vetor campo elétrico $\vec{E}$, a equação \eqref{2} é escrita como 
\begin{equation}\label{3}
\nabla\cdot\vec{E}(\vec{r})=\frac{\rho_{total}(\vec{r})}{\epsilon_{0}}=\frac{\rho_{ext}(\vec{r})+\rho_{pol}(\vec{r})}{\epsilon_{0}} ,
\end{equation}
onde $\epsilon_{0}$ é a permissividade elétrica no vácuo, $\rho_{total} = \rho_{ext} + \rho_{pol}$ é a densidade de carga de
polarização dada por $\rho P = -\nabla\cdot\vec{P}$ onde $\vec{P}$ é o vetor polarização elétrica. Para muitos materiais há uma
relação simples entre $\vec{E}$ e $\vec{D}$ dada por
\begin{equation}\label{4}
\vec{D} = \epsilon\vec{E} ,
\end{equation}
onde $\epsilon$ é a permissividade elétrica do material e no caso geral é uma função da posição e do tempo \eqref{2} Se considerarmos $\epsilon$ uma constante característica do material a equação \eqref{2} se torna
\begin{equation}\label{5}
\nabla\cdot\vec{E}(\vec{r})=\frac{\rho_{ext}(\vec{r})}{\epsilon} .
\end{equation}
As equações \eqref{1} e \eqref{5}, com condições de contorno apropriadas, determinam completamente o campo elétrico \eqref{3}
Apenas para sistemas suficientemente simétricos, é possível determinar o campo elétrico utilizando somente a lei de gauss, que
é a situação apresentada em todos os livros textos de eletromagnetismo [4]- [7]. A equação \eqref{1} %corrigir as referências aqui.
expressa o fato de que o campo eletrostático é conservativo, podendo, então, ser expresso como o gradiente de um campo escalar, i.e.,
\begin{equation}\label{6}
\vec{E}(\vec{r})= -\nabla\varphi(r) .
\end{equation}
Isso permite combinar as equações vetoriais \eqref{1} e \eqref{5} em uma única equação diferencial escalar. Utilizando \eqref{6} em \eqref{5} obtemos a equação de Poisson
\begin{equation}\label{7}
\nabla^{2}\varphi(r)=-\frac{\rho_{ext}(\vec{r})}{\epsilon}
\end{equation}
A princípio, o potencial elétrico pode ser determinado via equação de Poisson. Feito isso, podemos em seguida calcular o campo elétrico correspondente através da equação \eqref{6}. Esse foi o procedimento realizado na referência 1.






\end{multicols}
\end{document}
