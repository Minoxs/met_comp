%%PREAMBLE STARTS HERE%%

\documentclass[brazilian,12pt,a4paper]{article}
\usepackage[a4paper]{geometry}
\geometry{verbose,tmargin=1in,bmargin=0.8in,lmargin=1in,rmargin=0.8in}
\usepackage[brazil]{babel}
\usepackage[utf8]{inputenc}
\usepackage{graphicx}
\usepackage{amsmath}
\usepackage{amsfonts}
\usepackage{amssymb}
\usepackage{mathptmx}
\usepackage{multicol}

%\usepackage{multicolumns}
\pagestyle{empty}

%Title information%
\title{Comentários sobre o artigo "Eletrostática em sistemas coloidais carregados"\\
\large{Comments about the paper Electrostatics in charged colloidal systems}}
\author{A.V. Andrade-Neto\footnote{Endereço de correspondência: aneto@uefs.br.}\\ {\small Universidade Estadual de Feira de Santana, Departamento de Física, Feira de Santana, BA, Brasil}}
\date{\small{Recebido em 25 de Agosto, 2017. Revisado em 26 de Outubro, 2018. Aceito em 03 de Novembro, 2018.}}
%%%%%%%%%

%%PREAMBLE ENDS HERE%%
\begin{document}
\renewcommand{\abstractname}{}
%\begin{abstract}
\maketitle
\begin{abstract}
Comentam-se nessas notas resultados obtidos em trabalho publicado neste periódico sobre a eletrostática em
sistemais coloidais, Revista Brasileira de Ensino de Física \textbf{40}, e5408 (2018).\\
\textbf{Palavras-chave:} eletrostática, potencial blindado\\
\par
Some remarks about results of an article published in this journal, Revista Brasileira de Ensino de Física \textbf{40},
e5408 (2018), concerning electrostatics applied to colloidal systems are presented.\\
\textbf{Keywords:} electrostatic, screening.\\
\end{abstract}
\begin{multicols}{2}
No artigo intitulado “Eletrostática em sistemas co-
loidais carregados” [1], é apresentado “um estudo da
eletrostática aplicada aos sitemas coloidais”com o obje-
tivo de possibilitar “uma visão mais ampla dos conceitos
estudados em sala através de uma aplicação direta da
teoria”. Ainda que pertinente, por aplicar conceitos fa-
miliares da eletrostática a uma siuação de interesse, o
trabalho apresenta alguns problemas conceituais que dis-
cutiremos nessas notas. Apresentaremos uma crítica a
maneira como na Referência [1] são derivadas equações
da eletrostática para um problema específico (sistemas
coloidais) e como essas [equações (41) e (44) daquele tra-
balho] são interpretadas como uma possível nova lei de
Gauss. Indo direto ao ponto, são apresentadas expressões
que, segundo os autores, seriam “equações equivalentes
para a lei de Gauss, tanto na forma integral como na
forma diferencial”. Conforme mostraremos, trata-se ape-
nas das equações básicas da eletrostática escritas para
uma situação específica ou escritas de forma equivocada.
Vamos começar fazendo uma breve revisão da eletros-
tática, cujas leis fundamentais (equações de Maxwell)
para um meio dielétrico são dadas por
\begin{equation}\label{1}
\nabla\times\vec{E}(\vec{r}) = 0
\end{equation}
\begin{equation}\label{2}
\nabla\cdot\vec{D}(\vec{r}) = p_{ext}(\vec{r}) 
\end{equation}
onde $\vec{E}$ e $\vec{D}$ são, respectivamente, o campo elétrico e o vetor deslocamento
e $\rho_{ext}(\vec{r})$ é a densidade de carga externa. Em termos do vetor campo elétrico
$\vec{E}$, a equação (\ref{2}) é escrita como
\end{multicols}


\end{document}
