%%PREAMBLE STARTS HERE%%

\documentclass[brazilian,12pt,a4paper]{article}
\usepackage[a4paper]{geometry}
\geometry{verbose,tmargin=1in,bmargin=0.8in,lmargin=1in,rmargin=0.8in}
\usepackage[brazil]{babel}
\usepackage[utf8]{inputenc}
\usepackage{graphicx}
\usepackage{amsmath}
\usepackage{amsfonts}
\usepackage{amssymb}
\usepackage{mathptmx}

%\usepackage{multicolumns}
\pagestyle{empty}

%Title information%
\title{Comentários sobre o artigo "Eletrostática em sistemas coloidais carregados"\\
\large{Comments about the paper Electrostatics in charged colloidal systems}}
\author{A.V. Andrade-Neto\footnote{Endereço de correspondência: aneto@uefs.br.}\\ {\small Universidade Estadual de Feira de Santana, Departamento de Física, Feira de Santana, BA, Brasil}}
\date{\small{Recebido em 25 de Agosto, 2017. Revisado em 26 de Outubro, 2018. Aceito em 03 de Novembro, 2018.}}
%%%%%%%%%

%%PREAMBLE ENDS HERE%%
\begin{document}
\renewcommand{\abstractname}{}
%\begin{abstract}
\maketitle
\begin{abstract}
Comentam-se nessas notas resultados obtidos em trabalho publicado neste periódico sobre a eletrostática em
sistemais coloidais, Revista Brasileira de Ensino de Física \textbf{40}, e5408 (2018).\\
\textbf{Palavras-chave:} eletrostática, potencial blindado\\
\par
Some remarks about results of an article published in this journal, Revista Brasileira de Ensino de Física \textbf{40},
e5408 (2018), concerning electrostatics applied to colloidal systems are presented.\\
\textbf{Keywords:} electrostatic, screening.\\
\end{abstract}

\end{document}
