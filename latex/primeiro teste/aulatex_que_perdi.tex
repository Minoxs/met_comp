\documentclass[a4paper,12pt]{article}
\usepackage[brazil]{babel}
\usepackage[utf8x]{inputenc}
\usepackage{graphicx}
\usepackage{ucs}
\usepackage{amsmath}
\usepackage{amsfonts}
\usepackage{amssymb}

\usepackage{geometry}
\usepackage{hyperref}
%Código para cabeçalho
\usepackage{fancyhdr}      
\fancypagestyle{myfancy}
{
    \fancyhf{}
    \renewcommand{\headrulewidth}{0pt}
    \lfoot{Some Text}
    \rfoot{\thepage/\pageref{LastPage}}
}
%
\fancypagestyle{seite1}
{
   \fancyhf{}
   \renewcommand{\headrulewidth}{0.5pt}
   \lhead{some\\text}
   \chead{~\\even more text}
   \rhead{text\\\today}
   \rfoot{\thepage/\pageref{LastPage}}
}
\pagestyle{myfancy}
\geometry{left=3cm,right=3cm,top=1.6cm,bottom=3cm,headheight=0pt,headsep=10pt}

\title{Aula de Latex - Métodos Computacionais}
\author{Mino - IF-UFRGS}
\date{\today}

\begin{document}
\maketitle
\thispagestyle{myfancy}
\begin{abstract}
	Esse .tex está sendo usado para testes o3o
	\\Abstract é o resumo. A fonte é um pouco menor
	e as coisas ficam mais achatadas na página.
\end{abstract}

Olha que legal, o professor tá começando um novo documento kkkkkk
\\
\\Para escrever o TCC, provavelmente vamos usar Classe 'Book' ou 'Report'.
\\Podemos usar o detexify (está no moodle) para pegar o comando de símbolos gregos.
\\
\\Alguns testes:
\\
$f(x) = a*x + b$
\\
\begin{equation}
	\int_{0}^{\infty}f(x)dx
	\label{eq1}
\end{equation}
\\
\[\frac{1}{\psi} + \frac{1}{\gamma} = \frac{1}{\pi}\]
\\
\begin{align*}
\\&se
\\A &= B
\\ e&ntao
\\kB &= kA
\\k = c&onstante
\end{align*}
\\Isso que acabei de fazer tá bem feio e errado. Por favor fazer melhor.


\end{document}
